Con el propósito de cumplir con los objetivos planteados proponemos la realización de un modelo de regresión logística que permita describir las probabilidades de no-show y de cancelación como una función de las variables que caracterizan cada grupo de pacientes. Para esto la empresa ha puesto a disposición información correspondiente al registro de pacientes de los once centros médicos de la Red Salud UC durante el año 2015, la cual considera un total 45023 registros de citas médicas de 2277 pacientes. Este modelo permitirá caracterizar a cada conjunto de pacientes y determinar cuáles han sido los factores o combinación de estos que mejor explican el no-show y cancelación.

Un paso posterior a este consiste en la implementación de un modelo de optimización dinámica/estocástica que permita agendar en tiempo real las citas médicas de los pacientes, buscando maximizar el retorno esperado de la empresa considerando tanto el costo de sobre tiempo y ocio de los médicos y funcionarios, como también el tiempo de espera de los pacientes.

Un paso intermedio para darle mayor robustez al modelo es la inclusión de métodos de Estadística Bayesiana para una mejor estimación y actualización de las probabilidades de no-show y cancelación de cada paciente. Sin embargo es necesario evaluar la factibilidad de este paso dentro del alcance de este proyecto.
