Decimos que un evento de \textit{no-show} ocurre cuando un paciente agendado no asiste y no cancela su cita previamente. Una cancelación ocurre cuando un paciente contacta a la clínica y cancela sus citas agendadas. Este tipo de eventos provoca inconvenientes a la administración de la clínica, a los médicos y a los pacientes, y se pueden observar en la sobre y sub utilización de los médicos y mayores tiempos de espera de los pacientes, lo cual a su vez se traduce en menor satisfacción de los pacientes, menos productividad de la clínica y peor calidad del servicio \cite{daggy2010}. Se ha observado que los costos asociados a estos eventos pueden alcanzar cientos de miles de dólares al año \cite{alaeddini2015} y que el porcentaje de \textit{no-show} en clínicas de cuidado primario puede variar en el rango de 14 a 50 por ciento \cite{daggy2010}. Entonces, realizar una acertada predicción de la probabilidad de \textit{no-show} y cancelación es la piedra angular para cualquier sistema de agendamientos y estrategias de reducción de pacientes no atendidos \cite{alaeddini2015}.

Se ha visto que un diverso número de variables puede ser significativo para explicar el \textit{no-show}. Se ha encontrado que los pacientes con mayor tasa de \textit{no-show} tienden a ser “más jóvenes, no casados, sin seguro de salud, con problemas psicosociales y un historial de \textit{no-showing}” \cite{daggy2010}. Otras variables que se han incluido en otros estudios incluyen información geográfica, demográfica, socioeconómica y características de la consulta como: especialidad médica, día y tiempo hasta la cita programada, entre otros \cite{alaeddini2015, daggy2010,goldman1982}.

Los enfoques que existen para solucionar el problema de \textit{no-show} y cancelación de parte de la clínica se pueden dividir en dos. Uno de ellos intenta cambiar el comportamiento de los pacientes por medio de sanciones, educación, recordatorios y seguimiento, pero queda la pregunta sobre su aplicabilidad en distintos contextos, su efectividad para resolver el problema y si ésta puede ser mejorada considerando la probabilidad de \textit{no-show} y cancelación \cite{daggy2010}. El otro, intenta generar un mejor agendamiento de los pacientes, considerando la probabilidad de \textit{no-show} y cancelación, utilizando métodos como \textit{overbooking} o sobrecupos, \textit{short lead time} o tiempos de espera cortos,  y métodos de agendamiento óptimo.


El método de \textit{overbooking} típicamente se entiende como agendar un número fijo de pacientes adicionales para cada día basado en una única tasa de \textit{no-show} para la clínica. Una ventaja de este método es que es fácil de calcular y distintas políticas de \textit{overbooking} pueden llevar a agendas más óptimas \cite{berg2013}. Sin embargo, también se ha visto que este tipo de agenda finalmente incrementa el tiempo de espera de los pacientes, lo que se asocia también con un mayor tasa de \textit{no-show} \cite{daggy2010}.


Los métodos de \textit{short lead time}, buscan agendar citas dentro de los dos primeros días de solicitud. Sin embargo, no existe consenso sobre si este tipo de agendamiento es efectivo o no, y si lo fuera, sería muy dependiente del contexto particular de cada clínica \cite{daggy2010}.


Los métodos de agendamiento óptimo buscan ubicar a los pacientes de acuerdo a su probabilidad de \textit{no-show} individual maximizando el retorno esperado por la clínica, el cual incluye tanto el ingreso por cada cita como los costos asociados a las operaciones y al \textit{no-show} del paciente el día de la consulta. En este caso, es necesario una estimación robusta de la probabilidad de \textit{no-show} y de cancelación de cada paciente, y una metodología que permita determinar en tiempo real (en la duración de una llamada telefónica) cuál es la agenda óptima para cada paciente, que asegure al paciente tiempos de espera que sean razonables. En rigor los métodos de agendamiento óptimo pueden generar agendas de \textit{short lead time} y/o de \textit{overbooking}, pero no se limitan a ellas, sino que permiten adaptarse de acuerdo a los pacientes y los datos de la clínica.

Los textos que hoy se encuentran en la literatura respecto a este último tipo de agendamiento se pueden dividir en tres grupos: 1) Aquellos que se centran únicamente en la estimación y actualización de las tasas de \textit{no-show} y de cancelación \cite{alaeddini2011, alaeddini2015}; 2)  aquellos que asumen estas tasas conocidas, y las utilizan para estudiar y desarrollar métodos de agendamiento óptimo \cite{zeng2010, cayirli2003, liu2010, muthuraman2008}; y 3) aquellos que abordan ambos temas simultáneamente \cite{daggy2010, glowacka2009, goldman1982}. En general la metodología para estimar las probabilidades de \textit{no-show} y cancelación involucra el uso de modelos de regresión logística e inferencia Bayesiana \cite{alaeddini2015}.


En este trabajo nos centraremos en estudiar los modelos cuantitativos, de los cuales podemos encontrar diferentes categorías, las que en la literatura presentan diferentes ventajas y desventajas dependiendo del contexto en que se apliquen. Las principales categorías son:

\begin{itemize}
  \item Modelos de probabilidad homogénea: En este tipo de modelos de estimación de probabilidades, todos los pacientes tienen la misma probabilidad de \textit{no-show}. Es usado en gran parte de los estudios de \textit{overbooking} \cite{berg2013, zeng2010}. Son buenos modelos para plantear una primera política de \textit{overbooking} en caso de que no exista una, su mayor desventaja es no diferenciar entre pacientes ni entre grupos de pacientes.
  \item Modelos de probabilidad heterogénea: En este tipo de modelos de estimación de probabilidades, los pacientes tienen diferentes probabilidades de \textit{no-show}. Dentro de los modelos hay tres grandes categorías:
    \begin{itemize}
      \item Técnicas enfocadas en la población: En estos métodos se usa la información de la población como un conjunto de características con las cuales se estiman las probabilidades de \textit{no-show}, cancelación y \textit{show-up}. La regresión logística es uno de los métodos estadísticos más populares para estos casos  \cite{daggy2010,goldman1982}. También, en algunos casos se han usado modelos \textit{tree-based} y \textit{rule-based} para incrementar la homogeneidad de los subconjuntos de pacientes que se forman. Su ventaja es que proveen una buena estimación inicial para los pacientes, sin embargo, no diferencian entre los diferentes individuos y por lo tanto no se pueden actualizar efectivamente, especialmente usando pequeños grupos de datos. Otro problema que se presenta con estos métodos, es que agregar nuevos datos tiene un efecto insignificante en el resultado, especialmente cuando la base de datos inicial es mucho más grande en comparación con los nuevos datos \cite{alaeddini2015}.
      \item Técnicas enfocadas en el individuo: se basan principalmente en series de tiempo y metodos de suavizamiento, los cuales se usan para predecir las probabilidades requeridas. Sus ventajas radican en su rapidez de estimación y su efectividad para predecir el comportamiento individual, además de trabajar muy bien con bases de datos pequeñas,  sin embargo, su mayor problema radica en que no proveen una confiable estimación inicial. La principal razón de esto es que estos métodos usualmente usan una función de los datos pasados para estimar la probabilidad de un evento futuro. Pero el estado inicial en donde estamos no está disponible en la historia pasada, por lo cual la función no se puede aplicar en estos casos y esto provoca que por lo general se usa un valor inicial aleatorio \cite{alaeddini2015}.
      \item Técnicas híbridas: Son modelos que combinan los modelos anteriores, superando sus problemas y aprovechando sus ventajas. Usualmente consisten en una  regresión logística (enfoque de población)  y  estadística Bayesiana (enfoque individuo) en conjunto \cite{alaeddini2015}.
    \end{itemize}
\end{itemize}
