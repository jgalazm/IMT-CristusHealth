La Red de Salud UC Christus ( o Red Salud UC) es una red de atención médica privada en Chile compuesta por una asociación entre la Pontificia Universidad Católica de Chile y Christus Health, una de las diez instituciones de salud más grandes de Estados Unidos. Ésta cuenta con once centros médicos y más de mil médicos y cuatro mil trabajadores \cite{saluducAbout} con una oferta total de 65 especialidades médicas, incluyendo sub especialidades y apoyo médico en disciplinas como fonoaudiología y kinesiología.

El contacto con la empresa se ha realizado a través de Daniel Capurro quien desempeña como médico, Chief Medical Information Officer (CMIO) en la Red de Salud UC Christus y profesor asistente de la Facultad de Medicina de la Pontificia Universidad Católica de Chile. Luego de discutir en conjunto, se concluyó que uno de los problemas relevantes para la Red Salud UC Christus en la actualidad corresponde a la inasistencia de los pacientes a las citas programadas para consulta médica y la cancelación de estas mismas: el \textit{no-show} y la cancelación.

En efecto, a partir de información provista por la empresa, sólo durante el 2015 se observó que en promedio diariamente 107 citas presentaron \textit{no-show} (47\%) de un total de 228 reservadas, y que, además de las 228, 80 adicionales fueron canceladas.

Dentro de las medidas que se encuentran implementadas para disminuir el \textit{no-show} una consiste en realizar de uno a tres recordatorios telefónicos. El costo por llamada se estima en \$1.300, y considerando que en promedio el 2015 se atendieron 228 pacientes diarios, se estima un gasto diario promedio de hasta \$889.200, es decir MM\$324.558 anuales para la empresa.

Otra medida utilizada es la asignación de sobre cupos u \textit{overbooking}. Actualmente se le permite a cada médico decidir de acuerdo su agenda y experiencia cuántos cupos asignar para cada día por sobre el máximo posible por su horario y el de los demás. No existe una política general al respecto ni una estimación cuantitativa del costo o beneficio asociado.

Por esta razón, presentamos como objetivo en este trabajo el estudio, desarrollo y aplicación de una metodología precisa y cuantitativa que permita caracterizar y mejorar la situación actual de la empresa respecto al problema del \textit{no-show}, considerando tanto el beneficio de la empresa como el bienestar de los pacientes.
